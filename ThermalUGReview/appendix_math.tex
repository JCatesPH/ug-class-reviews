
\newpage
\renewcommand\thesubsection{\Alph{subsection}}

\phantomsection
\addcontentsline{toc}{section}{Math Appendix}
\section*{Math Appendix}


\subsection{Partial Derivative Relations}
\label{subsec::partials}

If you, like myself, find the derivative reduction process confusing, then I recommend reading the section in Appendix A of Callen called "Some Relations Involving Partial Derivatives." I summarize with the most useful relations below.

Begin with general function $\psi(x,y,z)$, and you can use the following relations between the partial derivatives:

\[
    \bigg(\frac{\partial y}{\partial x} \bigg)_{\psi,z} = \frac{-\Big(\frac{\partial \psi}{\partial x}\Big)_{y,z}} {\Big(\frac{\partial \psi}{\partial y}\Big)_{x,z}}   
\]
With similar expressions for the partials in the other derivatives. This is used to bring variable (usually S) out of the constants.

\[
    \bigg(\frac{\partial x}{\partial y} \bigg)_{\psi,z} = \frac{1} {\Big(\frac{\partial y}{\partial x}\Big)_{\psi,z}} 
\]
This one is the most elementary looking, and it is usually used to bring S or V to the numerator.

\[
    \bigg(\frac{\partial y}{\partial x} \bigg)_{\psi,z} = \frac{\Big(\frac{\partial y}{\partial u}\Big)_{\psi,z}} {\Big(\frac{\partial x}{\partial u}\Big)_{\psi,z}}
\]
Can be used to put T into the derivatives when using the standard set of derivatives.