%%%%%%%%%%%%%%%%%%%%%%%%%%%%%%%%%%%%%%%%%%%%%%%%%%
\chapter{The Third Test}

% PUT TOPICS HERE LATER


%--------------------------------------------%
\subsection{Economic Interpretation}
% 5.2 is on this.


%--------------------------------------------%
\section{Sensitivity Analysis}

%--------------------------------------------%
\subsection{Change to FILL}
%--------------------------------------------%
\subsection{Change to FILL}
%--------------------------------------------%
\subsection{Change to FILL}


%--------------------------------------------%
\subsection{Change to Right-Hand Side}
The basis $B$ remains a valid basis. The task in this case is checking the \textit{feasibility} of the new solution. If the solution remains optimal, then $B$ remains the optimal basis as $y$ is unchanged. For the change $\Vec{b}\rightarrow \Vec{b}^*$, the feasibility can be checked with the inequality
\begin{equation}
    \Vec{A}_B^{-1} \Vec{b}^* \geq \Vec{0}
\end{equation}



%--------------------------------------------%
\section{Dual Simplex Method}
The dual problem is being solved implicitly when performing the Simplex Method described before. Now we can take advantage of that.


%--------------------------------------------%
\subsection{Dual Feasibility}
Suppose $\Bar{c}_j \geq 0 \;\forall\; j\in N$ for a basis $B$. The basis $B$ is then called \textbf{dual feasible}. This is because the corresponding dual solution $\Vec{y}$ satisfies \begin{equation}
    0\geq\Bar{\Vec{c}}=\Vec{c}-\Vec{A}^T\Vec{y}
\end{equation}

\begin{theorem}
If $B$ is both feasible and dual feasible, then $B$ is an optimal basis.
\end{theorem}



%--------------------------------------------%
\section{Integer Programming Problems}
There are lots of problems that require optimizing over integer variables.


%--------------------------------------------%
\subsection{The Assignment Problem}
The task is assigning resources to tasks in an optimal way.




%--------------------------------------------%
\section{Geometry of Integer Programming}

