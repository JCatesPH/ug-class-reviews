\section{Grand-Canonical Formalism}

\emph{Kittel \& Kroemer: Chapter 5}
\begin{itemize}
    \item Know the Gibbs Sum \& Gibbs Factor
    \item Use thermal, chemical reservoir
    \item Construct Grand-Canonical partition function and compute expectation values
\end{itemize}

The Grand-Canonical Formalism is used to describe the states of a system in equilibrium with a thermal and chemical reservoir. The same arguments from the Canonical Formalism carries over.

\textbf{Gibbs Factor:} Gives the relative probability of a state $s$ with occupation $N_s$ and energy $\epsilon_s$ by
\[
\exp{[(N_s\mu-\epsilon_s)/\tau]}
\]

\textbf{Gibbs Sum:} (OR Grand-Canonical Partition Function) sums all of the Gibbs Factor so one can calculate the absolute probability of a given state.
\[
\mathcal{Z} = \sum_{all} \exp{[(N_s\mu-\epsilon_s)/\tau]}
\]
This gives the probability as
\[
\mathbb{P}(x) = \frac{e^{(N_x\mu-\epsilon_x)/\tau}}{\sum_{all} e^{(N_s\mu-\epsilon_s)/\tau}}
\]
The expectation values are calculated in the usual way.

\textbf{Absolute Activity:} Sometimes useful to use this lambda notation.
\[
\lambda \equiv e^{\mu/\tau}
\]