\section{Fermi-Dirac \& Bose-Einstein Distributions}

\emph{Kittel \& Kroemer: Chapter 6}
\begin{itemize}
    \item How to construct from Grand Canonical Formalism
    \item Will be given distribution, energy function, Fermi energy, etc. if needed
\end{itemize}
NOTE: Both distributions tend towards the same result in the high $\tau$ limit. This describes an "ideal gas", and it is the \textbf{Classical Distribution Function:}
\[
f(\epsilon) \approx e^{(\mu-\epsilon)/\tau} = \lambda e^{-\epsilon/\tau}
\]
\subsection{Fermi-Dirac Distribution}

Describes the distribution of fermions over energy states in a system. Fermions are 1/2 integer spin and obey Pauli Exclusion Principle (Either 0 or 1 particle occupying each state). The expectation value of the "Thermal Average Occupancy" $\langle N(\epsilon) \rangle$ in the Grand-Canonical Formalism gives the \textbf{Fermi-Dirac Distribution:}
\[
\langle N(\epsilon) \rangle = f_{FD}(\epsilon) = \frac{1}{e^{(\epsilon - \mu)/\tau}+1}
\]

\textbf{Fermi Level:} $\mu$ in this equation is often referred to by this name in Solid State Physics.

\textbf{Fermi Energy:} $\epsilon_F$ is the highest occupied energy value at $\tau=0$




\subsection{Bose-Einstein Distribution}

Describes the distribution of bosons over energy states in a system. Bosons have integer spin and can occupy the same state without limit. The "Thermal Average Occupancy" $\langle N(\epsilon) \rangle$ in the Grand-Canonical Formalism for bosons gives the \textbf{Bose-Einstein Distribution:}
\[
\langle N(\epsilon) \rangle = f_{BE}(\epsilon) = \frac{1}{e^{(\epsilon - \mu)/\tau}-1}
\]
