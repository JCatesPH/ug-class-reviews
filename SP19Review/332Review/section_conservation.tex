
\section{Conservation in Electrodynamics}

\emph{Noether's Theorem} states, \emph{If there is a conservation law, the action is stationary under an infinitesimal transformation in an appropriate variable.} Which can be stated as the converse, \emph{If the action is stationary under an infinitesimal transformation, there is a corresponding conservation law.}\footnote{Schwinger et al. \emph{Classical Electrodynamics}}

\begin{table}[!hbtp]
    \centering
    \begin{tabular}{c c}
        \toprule
        Transformed Variable  &  Conserved Quantity \\
        \toprule
        Time                  &  Energy             \\
        Space                 &  Momentum           \\
        Rotation              &  Angular Momentum   \\
        Gauge change          &  Charge             \\
    \end{tabular}
    \caption{A table of conserved quantities and the variable that implies them. The left column is the variable undergoing an infinitesimal transformation, and the right column is the implied conservation law.}
    \label{tab:conserve}
\end{table}

%\subsection{Conservation of Charge}
% FILL
% Div J = -d rho /dt


\subsection{Conservation of Energy}
This is a very consequential element of Griffiths' treatment of electrodynamics. The Energy for Electrostatics and Magnetostatics have been derived to be:\footnote{Note that these integrals are over all space, not the charge distribution.}

\begin{align*}
    W_e &= \frac{\epsilon_0}{2} \int E^2 d\tau \\
    W_m &= \frac{\mu_0}{2} \int H^2 d\tau
\end{align*}

The energy density can be written in two ways in vacuum:

\[
u = \frac{1}{2} \Big( \epsilon_0 E^2 + \frac{1}{\mu_0} B^2 \Big)  \text{    OR    } u = \frac{1}{2} \Big( \epsilon_0 E^2 + \mu_0 H^2 \Big)
\]
and in \emph{linear} media it becomes:
\[
u = \frac{1}{2} \Big( \textbf{E} \cdot \textbf{D} + \textbf{H} \cdot \textbf{B} \Big)
\]